%% template.tex
%% from
%% bare_conf.tex
%% V1.4b
%% 2015/08/26
%% by Michael Shell
%% See:
%% http://www.michaelshell.org/
%% for current contact information.
%%
%% This is a skeleton file demonstrating the use of IEEEtran.cls
%% (requires IEEEtran.cls version 1.8b or later) with an IEEE
%% conference paper.
%%
%% Support sites:
%% http://www.michaelshell.org/tex/ieeetran/
%% http://www.ctan.org/pkg/ieeetran
%% and
%% http://www.ieee.org/

%%*************************************************************************
%% Legal Notice:
%% This code is offered as-is without any warranty either expressed or
%% implied; without even the implied warranty of MERCHANTABILITY or
%% FITNESS FOR A PARTICULAR PURPOSE!
%% User assumes all risk.
%% In no event shall the IEEE or any contributor to this code be liable for
%% any damages or losses, including, but not limited to, incidental,
%% consequential, or any other damages, resulting from the use or misuse
%% of any information contained here.
%%
%% All comments are the opinions of their respective authors and are not
%% necessarily endorsed by the IEEE.
%%
%% This work is distributed under the LaTeX Project Public License (LPPL)
%% ( http://www.latex-project.org/ ) version 1.3, and may be freely used,
%% distributed and modified. A copy of the LPPL, version 1.3, is included
%% in the base LaTeX documentation of all distributions of LaTeX released
%% 2003/12/01 or later.
%% Retain all contribution notices and credits.
%% ** Modified files should be clearly indicated as such, including  **
%% ** renaming them and changing author support contact information. **
%%*************************************************************************


% *** Authors should verify (and, if needed, correct) their LaTeX system  ***
% *** with the testflow diagnostic prior to trusting their LaTeX platform ***
% *** with production work. The IEEE's font choices and paper sizes can   ***
% *** trigger bugs that do not appear when using other class files.       ***                          ***
% The testflow support page is at:
% http://www.michaelshell.org/tex/testflow/

\documentclass[conference,final,]{IEEEtran}
% Some Computer Society conferences also require the compsoc mode option,
% but others use the standard conference format.
%
% If IEEEtran.cls has not been installed into the LaTeX system files,
% manually specify the path to it like:
% \documentclass[conference]{../sty/IEEEtran}





% Some very useful LaTeX packages include:
% (uncomment the ones you want to load)


% *** MISC UTILITY PACKAGES ***
%
%\usepackage{ifpdf}
% Heiko Oberdiek's ifpdf.sty is very useful if you need conditional
% compilation based on whether the output is pdf or dvi.
% usage:
% \ifpdf
%   % pdf code
% \else
%   % dvi code
% \fi
% The latest version of ifpdf.sty can be obtained from:
% http://www.ctan.org/pkg/ifpdf
% Also, note that IEEEtran.cls V1.7 and later provides a builtin
% \ifCLASSINFOpdf conditional that works the same way.
% When switching from latex to pdflatex and vice-versa, the compiler may
% have to be run twice to clear warning/error messages.






% *** CITATION PACKAGES ***
%
%\usepackage{cite}
% cite.sty was written by Donald Arseneau
% V1.6 and later of IEEEtran pre-defines the format of the cite.sty package
% \cite{} output to follow that of the IEEE. Loading the cite package will
% result in citation numbers being automatically sorted and properly
% "compressed/ranged". e.g., [1], [9], [2], [7], [5], [6] without using
% cite.sty will become [1], [2], [5]--[7], [9] using cite.sty. cite.sty's
% \cite will automatically add leading space, if needed. Use cite.sty's
% noadjust option (cite.sty V3.8 and later) if you want to turn this off
% such as if a citation ever needs to be enclosed in parenthesis.
% cite.sty is already installed on most LaTeX systems. Be sure and use
% version 5.0 (2009-03-20) and later if using hyperref.sty.
% The latest version can be obtained at:
% http://www.ctan.org/pkg/cite
% The documentation is contained in the cite.sty file itself.






% *** GRAPHICS RELATED PACKAGES ***
%
\ifCLASSINFOpdf
  % \usepackage[pdftex]{graphicx}
  % declare the path(s) where your graphic files are
  % \graphicspath{{../pdf/}{../jpeg/}}
  % and their extensions so you won't have to specify these with
  % every instance of \includegraphics
  % \DeclareGraphicsExtensions{.pdf,.jpeg,.png}
\else
  % or other class option (dvipsone, dvipdf, if not using dvips). graphicx
  % will default to the driver specified in the system graphics.cfg if no
  % driver is specified.
  % \usepackage[dvips]{graphicx}
  % declare the path(s) where your graphic files are
  % \graphicspath{{../eps/}}
  % and their extensions so you won't have to specify these with
  % every instance of \includegraphics
  % \DeclareGraphicsExtensions{.eps}
\fi
% graphicx was written by David Carlisle and Sebastian Rahtz. It is
% required if you want graphics, photos, etc. graphicx.sty is already
% installed on most LaTeX systems. The latest version and documentation
% can be obtained at:
% http://www.ctan.org/pkg/graphicx
% Another good source of documentation is "Using Imported Graphics in
% LaTeX2e" by Keith Reckdahl which can be found at:
% http://www.ctan.org/pkg/epslatex
%
% latex, and pdflatex in dvi mode, support graphics in encapsulated
% postscript (.eps) format. pdflatex in pdf mode supports graphics
% in .pdf, .jpeg, .png and .mps (metapost) formats. Users should ensure
% that all non-photo figures use a vector format (.eps, .pdf, .mps) and
% not a bitmapped formats (.jpeg, .png). The IEEE frowns on bitmapped formats
% which can result in "jaggedy"/blurry rendering of lines and letters as
% well as large increases in file sizes.
%
% You can find documentation about the pdfTeX application at:
% http://www.tug.org/applications/pdftex





% *** MATH PACKAGES ***
%
%\usepackage{amsmath}
% A popular package from the American Mathematical Society that provides
% many useful and powerful commands for dealing with mathematics.
%
% Note that the amsmath package sets \interdisplaylinepenalty to 10000
% thus preventing page breaks from occurring within multiline equations. Use:
%\interdisplaylinepenalty=2500
% after loading amsmath to restore such page breaks as IEEEtran.cls normally
% does. amsmath.sty is already installed on most LaTeX systems. The latest
% version and documentation can be obtained at:
% http://www.ctan.org/pkg/amsmath





% *** SPECIALIZED LIST PACKAGES ***
%
%\usepackage{algorithmic}
% algorithmic.sty was written by Peter Williams and Rogerio Brito.
% This package provides an algorithmic environment fo describing algorithms.
% You can use the algorithmic environment in-text or within a figure
% environment to provide for a floating algorithm. Do NOT use the algorithm
% floating environment provided by algorithm.sty (by the same authors) or
% algorithm2e.sty (by Christophe Fiorio) as the IEEE does not use dedicated
% algorithm float types and packages that provide these will not provide
% correct IEEE style captions. The latest version and documentation of
% algorithmic.sty can be obtained at:
% http://www.ctan.org/pkg/algorithms
% Also of interest may be the (relatively newer and more customizable)
% algorithmicx.sty package by Szasz Janos:
% http://www.ctan.org/pkg/algorithmicx




% *** ALIGNMENT PACKAGES ***
%
%\usepackage{array}
% Frank Mittelbach's and David Carlisle's array.sty patches and improves
% the standard LaTeX2e array and tabular environments to provide better
% appearance and additional user controls. As the default LaTeX2e table
% generation code is lacking to the point of almost being broken with
% respect to the quality of the end results, all users are strongly
% advised to use an enhanced (at the very least that provided by array.sty)
% set of table tools. array.sty is already installed on most systems. The
% latest version and documentation can be obtained at:
% http://www.ctan.org/pkg/array


% IEEEtran contains the IEEEeqnarray family of commands that can be used to
% generate multiline equations as well as matrices, tables, etc., of high
% quality.




% *** SUBFIGURE PACKAGES ***
%\ifCLASSOPTIONcompsoc
%  \usepackage[caption=false,font=normalsize,labelfont=sf,textfont=sf]{subfig}
%\else
%  \usepackage[caption=false,font=footnotesize]{subfig}
%\fi
% subfig.sty, written by Steven Douglas Cochran, is the modern replacement
% for subfigure.sty, the latter of which is no longer maintained and is
% incompatible with some LaTeX packages including fixltx2e. However,
% subfig.sty requires and automatically loads Axel Sommerfeldt's caption.sty
% which will override IEEEtran.cls' handling of captions and this will result
% in non-IEEE style figure/table captions. To prevent this problem, be sure
% and invoke subfig.sty's "caption=false" package option (available since
% subfig.sty version 1.3, 2005/06/28) as this is will preserve IEEEtran.cls
% handling of captions.
% Note that the Computer Society format requires a larger sans serif font
% than the serif footnote size font used in traditional IEEE formatting
% and thus the need to invoke different subfig.sty package options depending
% on whether compsoc mode has been enabled.
%
% The latest version and documentation of subfig.sty can be obtained at:
% http://www.ctan.org/pkg/subfig




% *** FLOAT PACKAGES ***
%

%\usepackage{fixltx2e}
% fixltx2e, the successor to the earlier fix2col.sty, was written by
% Frank Mittelbach and David Carlisle. This package corrects a few problems
% in the LaTeX2e kernel, the most notable of which is that in current
% LaTeX2e releases, the ordering of single and double column floats is not
% guaranteed to be preserved. Thus, an unpatched LaTeX2e can allow a
% single column figure to be placed prior to an earlier double column
% figure.
% Be aware that LaTeX2e kernels dated 2015 and later have fixltx2e.sty's
% corrections already built into the system in which case a warning will
% be issued if an attempt is made to load fixltx2e.sty as it is no longer
% needed.
% The latest version and documentation can be found at:
% http://www.ctan.org/pkg/fixltx2e


%\usepackage{stfloats}
% stfloats.sty was written by Sigitas Tolusis. This package gives LaTeX2e
% the ability to do double column floats at the bottom of the page as well
% as the top. (e.g., "\begin{figure*}[!b]" is not normally possible in
% LaTeX2e). It also provides a command:
%\fnbelowfloat
% to enable the placement of footnotes below bottom floats (the standard
% LaTeX2e kernel puts them above bottom floats). This is an invasive package
% which rewrites many portions of the LaTeX2e float routines. It may not work
% with other packages that modify the LaTeX2e float routines. The latest
% version and documentation can be obtained at:
% http://www.ctan.org/pkg/stfloats
% Do not use the stfloats baselinefloat ability as the IEEE does not allow
% \baselineskip to stretch. Authors submitting work to the IEEE should note
% that the IEEE rarely uses double column equations and that authors should try
% to avoid such use. Do not be tempted to use the cuted.sty or midfloat.sty
% packages (also by Sigitas Tolusis) as the IEEE does not format its papers in
% such ways.
% Do not attempt to use stfloats with fixltx2e as they are incompatible.
% Instead, use Morten Hogholm'a dblfloatfix which combines the features
% of both fixltx2e and stfloats:
%
% \usepackage{dblfloatfix}
% The latest version can be found at:
% http://www.ctan.org/pkg/dblfloatfix




% *** PDF, URL AND HYPERLINK PACKAGES ***
%
%\usepackage{url}
% url.sty was written by Donald Arseneau. It provides better support for
% handling and breaking URLs. url.sty is already installed on most LaTeX
% systems. The latest version and documentation can be obtained at:
% http://www.ctan.org/pkg/url
% Basically, \url{my_url_here}.




% *** Do not adjust lengths that control margins, column widths, etc. ***
% *** Do not use packages that alter fonts (such as pslatex).         ***
% There should be no need to do such things with IEEEtran.cls V1.6 and later.
% (Unless specifically asked to do so by the journal or conference you plan
% to submit to, of course. )



%% BEGIN MY ADDITIONS %%


\usepackage{longtable,booktabs}

\usepackage[unicode=true]{hyperref}

\hypersetup{
            pdfborder={0 0 0},
            breaklinks=true}
\urlstyle{same}  % don't use monospace font for urls

% Pandoc toggle for numbering sections (defaults to be off)
\setcounter{secnumdepth}{5}

% Pandoc syntax highlighting


% Pandoc header
\usepackage[none]{hyphenat}
\usepackage{longtable}
\pagestyle{plain}
\raggedbottom
\let\origtable\table
\let\origendtable\endtable
\renewenvironment{table}[1][2] { \expandafter\origtable\expandafter[!ht] } { \origendtable }
\let\origfigure\figure
\let\origendfigure\endfigure
\renewenvironment{figure}[1][2] { \expandafter\origfigure\expandafter[!ht] } { \origendfigure }
\usepackage{caption}
\captionsetup[table]{textfont={it}, labelfont={bf}, singlelinecheck=false, labelsep=newline}

\providecommand{\tightlist}{%
  \setlength{\itemsep}{0pt}\setlength{\parskip}{0pt}}

%% END MY ADDITIONS %%


\hyphenation{op-tical net-works semi-conduc-tor}

\begin{document}
%
% paper title
% Titles are generally capitalized except for words such as a, an, and, as,
% at, but, by, for, in, nor, of, on, or, the, to and up, which are usually
% not capitalized unless they are the first or last word of the title.
% Linebreaks \\ can be used within to get better formatting as desired.
% Do not put math or special symbols in the title.
\title{}

% author names and affiliations
% use a multiple column layout for up to three different
% affiliations

\author{
}

% conference papers do not typically use \thanks and this command
% is locked out in conference mode. If really needed, such as for
% the acknowledgment of grants, issue a \IEEEoverridecommandlockouts
% after \documentclass

% for over three affiliations, or if they all won't fit within the width
% of the page, use this alternative format:
%
%\author{\IEEEauthorblockN{Michael Shell\IEEEauthorrefmark{1},
%Homer Simpson\IEEEauthorrefmark{2},
%James Kirk\IEEEauthorrefmark{3},
%Montgomery Scott\IEEEauthorrefmark{3} and
%Eldon Tyrell\IEEEauthorrefmark{4}}
%\IEEEauthorblockA{\IEEEauthorrefmark{1}School of Electrical and Computer Engineering\\
%Georgia Institute of Technology,
%Atlanta, Georgia 30332--0250\\ Email: see http://www.michaelshell.org/contact.html}
%\IEEEauthorblockA{\IEEEauthorrefmark{2}Twentieth Century Fox, Springfield, USA\\
%Email: homer@thesimpsons.com}
%\IEEEauthorblockA{\IEEEauthorrefmark{3}Starfleet Academy, San Francisco, California 96678-2391\\
%Telephone: (800) 555--1212, Fax: (888) 555--1212}
%\IEEEauthorblockA{\IEEEauthorrefmark{4}Tyrell Inc., 123 Replicant Street, Los Angeles, California 90210--4321}}




% use for special paper notices
%\IEEEspecialpapernotice{(Invited Paper)}




% make the title area
\maketitle

% As a general rule, do not put math, special symbols or citations
% in the abstract
\begin{abstract}

\end{abstract}

% keywords

% use for special paper notices



% make the title area
\maketitle

% no keywords

% For peer review papers, you can put extra information on the cover
% page as needed:
% \ifCLASSOPTIONpeerreview
% \begin{center} \bfseries EDICS Category: 3-BBND \end{center}
% \fi
%
% For peerreview papers, this IEEEtran command inserts a page break and
% creates the second title. It will be ignored for other modes.
\IEEEpeerreviewmaketitle


\hypertarget{method-and-materials}{%
\section{Method and Materials}\label{method-and-materials}}

A sample of texts is required to classify the binary sentiment of a movie script or Twitter Tweets.
The movie script text samples represent dimensional space for the set task.
The words in the language determine the sentiment of the text sample.
A supervised machine learning algorithm, a deep neural network known as a C-LSTM, which is a hybrid neural network including convolutional and long-short-term-memory layers, makes it possible to extract the sentiment from the text alone.
Applying adaptable parameters and structures within the C-LSTM layers provides an architecture capable of performing text classification tasks reliably across different dimensions of texts.
All modeling efforts are captured in the following order:

\begin{enumerate}
\def\labelenumi{\arabic{enumi})}
\tightlist
\item
  Text Pre-Processing \& Sentiment Labeling
\item
  Hyper-Tuning of the PTCN
\item
  Best Parameters PTCN Model 5-fold Cross-validation
\item
  Capture Embedding Features from the Predictions
\end{enumerate}

A series of natural language processing techniques are required to supervise the training of the PTCN to classify the text sentiment.
The text samples contain noise, which is in the form of stop words, uppercasing, special characters, NA values, punctuation, numbers, and whitespace.
The removal of noise from the text samples helps mitigate the C-LSTM, recognizing patterns within the texts that create bias predictions.
Text samples with mitigated noise are not the only requirement for training the PTCN.
A classification of each of the text samples' binary sentiment is needed.
A label of sentiment for each text sample is completed by calculating the polarity of each word in a text sample and concluding the overall polarity score.
The overall polarity score determines the classification of sentiment for each text based on a conditional threshold of above or below a neutral value of 0.
If the polarity is above 0, the classification label is positive for that text sample.
If the value is below 0, the classification label is negative.
Each polarity measure is converted into a binary label depending on the conditional threshold.

The PTCN is a deep learning algorithm that automatically extracts features from an input vector with adaptable kernel initializers.
An additional batch normalization layer and dropout layer are implemented to help avoid bias in the model's predictions. Deep neural networks transform inputs over extensive credit assignment pathways (CAPs).
The more complicated the dimensions of input, then the more complex the CAPs.
The labeled sentiment text samples are balanced into an equal distribution to confirm that the features extracted throughout the transformation of the text inputs are significant.
The C-LSTM will be trained on an equal amount of random positive sentiment text samples and random negative sentiment text samples.
For the hyper-tuning model, the text samples are embedded into the layers of the C-LSTM in the form of a tokenized word vector.
The space of the tokenized word vector is the representation of the positive and sentiment movie scripts that the model will extract features to form an accurate prediction.
A list of input samples is tokenized in sequence to form numerical values for each represented word across the texts.
Each text is transformed into a numerical sequence of tokenized words in the form of a vector.
The newly formed word vector spaces are attached to their designated sentiment label.
The lists of input vectors and output scalars require further transformation to embed with the C-LSTM. The text inputs are padded in order to make each text the same size.
The list of padded inputs is transformed into a matrix. The outputs are converted into categorical variables for each sample transforming from a list of scalars into a matrix of binary responses.
The complete sample of text and sentiment labels are split for training and evaluation datasets.

For the best model, a small variation is required in the processing of the sentiment labeled text samples.
Texts processing techniques all remain the same, but the sampling for training and validation sets is changed to incorporate 5-fold cross-validation.
A fixed set of 100 samples is reserved for evaluating the best model's folds.
After preprocessing the data for the hyper-tuning and best model, the data is embedded into the C-LSTM layers and transformed through the stacks.
The inputs are embedded based on their dimensions, including the max features, embedding dimensions, and max length of the inputs.
The inputs are piped through a series of layers in the model. 1-dimensional convolutional layers are set with a series of elements to control various parameters of the convolutions.
The input samples are large vectors with high sparsity requiring that the layers be allowed to explore but in a controlled manner.
A series of convolutional layers are stacked and activated using a leaky Relu function to promote the model identifying patterns in the inputs.
The model does not overfit because max-pooling layers are implemented at specific points in the architecture of the model.
To further mitigate overfitting.
The model is designed to pipe the inputs through a series of convolutional layers and then into a long-short-term-memory (LSTM) layer set with various parameter controls and activated with a leaky Relu function.
In the last layer, the inputs are piped into a fully connected layer activated using a sigmoid function.
The model is compiled using a binary cross-entropy function, optimized with a nadam function, and reporting results in set metrics.

The model is fit to the training dataset of inputs and outputs through a set number of epochs.
Deep learning models validate after each epoch.
A specific percentage of the training data is set for validation.
To ensure the model does not train further, then it can learn, early stopping parameters are utilized to automate a change in the learning rate of the optimizer function and stop the model if improvement plateaus after a set number of epochs.
The best weights are captured and set for the final model state that is saved for evaluation.
The evaluation dataset is tested using the fitted model and determines the model's ability to generalize to new inputs.
Hyper-tuning the model's parameters will ensure possible values to set for final modeling.
The final model is trained using 5-fold cross-validation to fully understand the predictive capabilities of the model.

To ensure that the model is understanding the language after the best model is captured and cross-validated.
The features embedded in the layers of the network are visualized to express the features of words that support or contradict the predicted sentiment classification.
Understanding the features helps determine the usefulness of the convolutions recognizing features from the word vector spaces.
The layers throughout the stack our essential to the performance of the PCTN.
The sequential deep learning model is constructed with a stack of different layers set with a number of parameters, including:

\begin{itemize}
\tightlist
\item
  Dropout rate,
\item
  Convolutional hidden nodes,
\item
  LSTM hidden nodes,
\item
  L1 regularization rate,
\item
  L2 regularization rate,
\item
  Batch size,
\item
  Input max length,
\item
  Max features,
\item
  Embedding dimensions,
\item
  Leaky Relu rate,
\item
  Kernel size,
\item
  Epochs,
\item
  Max-pooling size,
\item
  Learning rate, and
\item
  Validation split.
\end{itemize}

The first layer in the model is the embedding layer, which embeds the tokenized words into the network.
The word vector spaces reflect a commonly referred to method in natural language processing known as a bag of words method.
The convolutional and LSTM layers in the architecture are capable of extracting features automatically from the highly sparse dimensional space of the processed inputs.
The next layer the inputs are piped into is a 1-dimensional convolutional layer with only a 4th of the set convolutional hidden nodes.
The inputs are representations of long and highly sparse word vector spaces which makes the initialization of the model weights critically important.
The model requires a method to adjust to the high variance between samples.
The control of the kernels will help the model steadily learn from the features extracted from the text.

Using a variance scaling kernel initializer provides an \emph{``\ldots{} initializer that is capable of adapting its scale to the shape of weights.''} \emph{(R-Documentation, 2019)}.
The Convolutional layers initializer make the deep network adapt to the input weights.
It is important to regularize the kernels when utilizing an adaptive initializer.
Setting a duel regularization technique, which deploys L1 and L2 regularization helps mitigate high fluctuations while batching samples through the layers. L1 is Lasso Regression (LR).
It is a penalty term measuring the \emph{``. absolute value of the magnitude of the coefficient.''} \emph{(R Documentation, 2019)}.
L2 is Ridge Regression.
A penalty term measuring the \emph{``\ldots{} squared magnitude of coefficient.''} \emph{(R Documentation, 2019)}.
The main difference between the two methods is the penalty term. The 1-dimensional convolutional layers are set to inputs padded to the same shape.
Strides are set to 1L through the convolutions, which are an \emph{"\ldots{} an integer or list of a single integer, specifying the stride length of the convolution.} \emph{(R Documentation, 2019)}.
The convolutional layer is activated using a Leaky Relu function.
The leaky rely function allows for a \emph{``.small gradient when the unit is not active''}, providing \emph{``\ldots{} slightly higher flexibility to the model than traditional Relu.''} \emph{(R Documentation, 2019)}.
The first convolutional layer extracts lower level features from the inputs due to the decrease in the hidden nodes.
The reduction of the number of transformations provides control to the adaptable features initializing the weights.
A high kernel size provides the layer with the ability to transform the data at a higher level through the few transformations.
The second layer piped to is another 1-dimensional convolutional layer with the same parameters set but except the number of hidden nodes is set to 32.
Increasing the number of hidden nodes provides more transformations extracting higher level features from the inputs.
The second convolutional layer is activated using the leaky Relu function.
The next layer in the stack is another convolutional layer set at half the set number of hidden nodes.

To mitigate overfitting a max-pooling layer set to 4 is implemented.
Following is two more layers of 1-dimensional convolutional layers set to half the number of set hidden nodes and a 4th of the set hidden nodes.
All parameters are set the same as the prior convolutional layers.
A second max-pooling layer, batch normalization, and dropout layer are set to further mitigate overfitting.
The next layer is a LSTM layer with the set at 32 hidden nodes. Variance Scaling kernel initializers and L1 and L2 kernel regularizers are implemented.
The LSTM layer is activated using a Leaky Relu function.
A dropout layer set at 0.5 is implemented before the output layer. The output layer is activated using a sigmoid function.
The model is compiled using a loss function of binary cross-entropy.
The stochastic gradient descent (SGD) optimizer is set with a learning rate at a specific learning rate.
The learning rate will be adjusted to different rates during the hyper-tuning sessions to provide the best rate of learning for the final model.
The models will provide training, validation, and evaluation metrics in form of accuracy percentage.
The results will also provide the loss measure of the training, validation, and evaluation.

The embedding layers help understand the capabilities of the model.
The model's predictions can be explained and provide insight into the features supporting or contradicting the sentiment classification of a text sample based on the weighted measure.
The results include the predicted samples label, probability, and explanation fit.
Within the visual 20 features are set to express their weights to support or contradict the texts classification.

\end{document}


